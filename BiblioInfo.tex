% Filling in this bibliographic information facilitates the
% processing of this document.

% Insert linebreaks if necessary

% Just leave the posistion blank for any information that doesn't
% apply for your thesis, e.g. if your thesis doesn't have a subtitle:
%
% \newcommand{\thesisSubtitle}{}




% Abstract and titelpage

\firstAuthorFirstName{Leon}                % First author given name
\firstAuthorSurname{Hennings}                 % First author surname
\secondAuthorFirstName{Kamyar}              % Second author given name
\secondAuthorSurname{Sajjadi}                % Second author given name

\firstAuthorEmail{leon.hennings@gmail.com}              % First author's e-mail address
\secondAuthorEmail{kasa5879@student.su.se}             % Second author's e-mail address

\thesisTitle{Re-engineering the MediaSense Platform towards Resource Constrained Devices}                   % The title of the thesis
\thesisSubtitle{RPC-based Daemon Approach} % The subtitle of the thesis 

\thesisSubject{Computer and Systems Sciences}   % May be changed
                                                % to e.g. Human-Computer 
                                                % Interaction or other 
                                                % suitable for your thesis 
\thesisIsKind{Bachelor}                           % Change to Bachelor if suitable
\theYear{2013}
\thesisCred{15}                                 % Change to 15 if Bachelor
\thesisAdvisor{Jamie Walters}
\thesisAssistantAdvisor{}                       % Name of Assistant Advisor,
                                                % if you have one
\thesisExternalAdvisor{}                        % Name of External Advisor,
                                                % if you have one

\thesisReviewer{Iskra Popova}
\semester{Spring}
\swedishTitle{}

                                               % The abstract text comes
                                               % here. Not more than 300
                                               % words. No empty lines.                                     
\abstracttext{Computers are becoming pervasive in our society through the use of tablets, smartphones, computers embedded in home appliances and media devices. These devices are becoming more connected, creating an Internet of Things. The Internet of Things will make use of all collected data currently stored separately in devices and allow for a new type of immersive applications by utilizing users' data from several sources. 
MediaSense as an Internet of Things middleware allows for communication between devices and distributed sharing of context information. MediaSense in its current form it is not 
geared towards smaller, portable computers that are making up the Internet of Things. This project applies a design science methodology for re-engineering the platform to make it usable on ubiquitous devices. This thesis propose a redesign intended to create a shared background service, a so called daemon, of the MediaSense platform so the functionality of sharing and storing sensor data is shared among the applications using it. To implement this behavior, a type of inter-process communication had to be chosen to allow the applications to communicate with the platform. The Java implementation of Remote Procedure Call called Remote Method Invocation was chosen. The new version of MediaSense was designed and developed according to the requirements outlined by the lead developer of MediaSense. The redesign resulted in lower memory usage and less CPU time when running more than one application. 
This discovery will make Internet of Things middleware possible to use on ubiquitous devices and enable several immersive applications to run simultaneously on one device.}



\keywords{Immersive Participation, Context Awareness, Pervasive Computing, Ubiquitous Computing, Internet Of Things, Middleware, MediaSense, Distributed, Inter-Process Communication, Remote Method Invocation}


