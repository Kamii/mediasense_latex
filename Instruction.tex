%\part{Introduction}%
% Parts can be used to add a headline on the level above
% "Chapter". This is typically not needed in a Master Thesis. 

\chapter{Introduction}
\section{What's in this package?}
In this package you will find the following files:
\vspace{1em}
\begin{simplelist}
    \item \textbf{BiblioInfo.tex} - Contains all bibliographic
      information and information about the thesis. The author
      should add this information to the commands defined in this
      file. 
    \item \textbf{Example.tex} - An example to show the
      typesetting of all the used commands and environments. 
    \item \textbf{Instruction.tex} - This file.
    \item \textbf{Introduction.tex} - Tex file for the first
      chapter. 
    \item \textbf{Read.me} - Contains basic instructions for the
      package. 
    \item \textbf{References.bib} - An example of a reference
      file. 
    \item \textbf{Sunset.jpg} - The image in the example.
    \item \textbf{Sunset.eps} - The image in the example.
    \item \textbf{Fonts.pdf} - The image in the instruction.
    \item \textbf{Fonts.eps} - The image in the instruction.    
    \item \textbf{Thesis.blg, Thesis.aux, Thesis.bbl Thesis.log,
        Thesis.lot, Thesis.out Thesis.toc} - Help files. 
    \item \textbf{Thesis.pdf} - The PDF result Thesis.tex. 
    \item \textbf{Thesis.ps} - The PostScript result Thesis.tex. 
    \item \textbf{Thesis.dvi} - The DVI result Thesis.tex. 
    \item \textbf{Thesis.sty} - All the changes to and
      redefinitions of Book.cls have been collected in this
      file. It also contains the basic structure of the document
      along with all new commands and environments used in the
      template as well as all redefined commands and
      environments. This file defines the commands used in the
      front matter that create the half-title page, abstractpage,
      list of papers and dedication.    
    \item \textbf{Thesis.tex} - This is the main \TeX{} file. In
      this frame all other \TeX{} files are included. 
    \item \textbf{SU-Logga.eps} - The Stockholm university black
      and white logotype in EPS format. 
    \item \textbf{SU-Logga} - The Stockholm university black and
      white logotype in PDF format. 
    \item \textbf{ProblemsAndSolutions.tex} - File containing
      known problems and suggested solutions. 
    \item \textbf{captions.sty} - The used version of caption
      package. If you miss the latest version of captions use this
      file. 
\end{simplelist}

\section{Instructions}
The template consists of one main file, \emph{Thesis.tex}, and a
number of \TeX{} files which are included in the mainfile. The
thesis may be split into one or several chapter files which are to
be included in the main matter of the Thesis.tex
file. Introduction.tex has been included to illustrate this. All
the normal commands and environments defined in \LaTeX{} can be
used although some of them have been redefined for typographical
reasons. In addition to the thesis the author should edit the
BiblioInfo.tex file. Finally, Instruction.tex and Example.tex,
should be excluded from Thesis.tex. 

\subsection{New environments}
A number of list environments have been added to the
template. These are: 
\begin{tabbing}
\hspace{4cm}\=\\
enumerate-indent \> Same as enumerate (numbered list) but
indented\\ 
itemize-indent \> Same as itemize (bulleted list) but indented\\ 
romanlist \> Roman list (Roman numbers)\\
romanlist-indent \> Same as romanlist but indented\\
simplelist \> Simple list (no bullets or numbers)\\
simplelist-indent \> Same as simplelist but indented.
\end{tabbing}

\section{Important note}
This template may use the pdflatex package to produce the output
as a PDF file. The package requires that all images are in JPEG,
PNG or PDF format. EPS files can easily be converted to PDF files
using the \emph{ps2pdf} command in the terminal. E.g. ps2pdf
imagefile.eps imagefile.pdf. The Linux \emph{convert} command can
be used to convert other formats to PDF files. Note that it is not
necessary to use the \textit{pdflatex} package and that this
version of the template also produces DVI and PostScript files.    

If any of the used packages are missing or out of date in your
\LaTeX{} installation, the latest version can be downloaded from
http://www.ctan.org/. Packages such as geometry, caption, and many
others are typically made 
up of two files: a file with the extension .ins and another with
the extension .dtx. Run \LaTeX{} on the .ins file. 
This will extract a .sty file. Move the .sty file to a place where
your distribution can find it. Usually this is in your
.../localtexmf/tex/latex subdirectory (Windows or OS/2 users
should feel free to change the direction of the slashes). Refresh
your distributions file-name database. The command depends on the
\LaTeX{} distribution you use: te\TeX{}, fp\TeX{} \code{texhash};
web2c \code{maketexlsr}; Mik\TeX{} \code{initexmf -update-fndb} or
use the GUI. 
 
\subsection{Fonts}
PDF files created from \LaTeX{} usually don't include embedded
fonts. This is due to the fact that the default setting for the
distributions in TeX are configured to exclude the 14 basic fonts,
for example Times and Helvetica.

\section{Outline}
A comprehensive summary should include the following parts in the following order:
\begin{itemize}
    \item Title page.
    \item Abstract page.
    \item Table of Contents.
    \item Introduction/Background (the first chapter; the first page to be paginated using Arabic numerals).
    \item Chapter 1 \ldots n.
    \item References/Bibliography.
\end{itemize}
\vspace{1\baselineskip}
\noindent The front matter (the sequence of pages from the half-title page or the title page up to the table of contents) is never paginated. The sequence from Chapter 1 up to References/Bibliography is paginated using Arabic numerals. 
