Smartphones and mobile broadband allow access to the Internet anywhere at any time. There are many types of devices with sensors which generate data. As shown in \cite{chui2010internet}, sensors from different devices can be connected to each other using Internet Protocol (IP). This is resulting in an Internet of connected things, where each thing, whether human or machine can connect and communicate, sharing and digesting information, executing tasks and collaborating to realise massive immersive environments as described in \cite{tan2010future}. This new network, the Internet of Things, aims to seamlessly fuse people places and things across current communications platforms, realising immersive situations that are enabled through the collection of information from embedded sensors and respond by acting upon corresponding embedded actuators. 

Sensors embedded within physical environment range from simple sensors such as temperature, humidity, light intensity and occupation sensors, to location and Global Positioning System (GPS) sensors embedded in mobile devices, telephones and automobiles. Applying approaches such as sensor fusion allows us to emulate even higher level sensor information exposing information that is otherwise not directly observable. All the sensors collect data that developers can use to build applications that respond to the given data from the sensors. These could range from an application that can automatically regulate the heating in your home to an application that can tell you which way you should take to work according to the traffic on streets.

In order to make this kind of applications possible and realize the impending immersive paradigm, sensors of a device need to collect and share context information \cite{dey2001understanding}. Earlier research in the area explored the use of middleware systems for large scale information provisioning, these included both centralized and distributed approaches. Centralized approaches such as the IP Multimedia subsystem \cite{Kardeby:2010:UMF:1845879.1846331} and MQTT (Message Queue Telemetry Transport) \cite{HunkelerTS08}, are web services on the Internet, providing a point of connection for entities to provision context information on an Internet of Things. However, these approaches assume the complete availability and reliability systems which are susceptible to Domain Name System (DNS) errors, denial of service attacks and dynamic IP configuration issues. Centralized approaches create bottlenecks which affect real-time information sharing. Without real-time information the freshness and accuracy of the information cannot guaranteed, as shown in \cite{Walters437970}. This is important for an Internet of Things where real-time is key to creating Immersive Participation Environments. Furthermore, the scalability is limited when using a centralized approach \cite{Kanter539187}. The problems with scalability and the vulnerability of DNS makes centralised solutions suboptimal. 

Distributed approaches have been developed as alternative approaches as they have a leaner dependency on DNS, more scalable and are less prone to denial-of-service attack. One such distributed approach to Internet of Things middleware is MediaSense \cite{Kanter539187}, an active research project at Stockholm University. A device connected to MediaSense is responsible for receiving and storing the distributed context information as well as generating and sharing its own context information. Within the mediasense realization, applications are tied to the platform and are started and terminated with the platform. 

Every application has its own MediaSense instance and communicates with other devices through this instance. One thing that needs to be considered when developing middleware for the  Internet of Things is that devices in our everyday life has limited resources. The current way applications run on MediaSense, where every application needs its own instance of the platform makes it very resource inefficient, as shown in \ref{tab:test_table}.