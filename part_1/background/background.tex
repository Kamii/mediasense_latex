Around us today is a large assortment of devices whose sensors generate data. This data is mostly used locally or together with other devices in small systems. Smartphones and mobile broadband allow people to have access to the internet wherever they are. As shown in \cite{chui2010internet}, sensors from different devices can be connected to each other using Internet Protocol. This is resulting in an Internet of Connected things, where each thing, whether human or machine can connect and communicate, sharing and digesting information, executing tasks and collaborating to realise massive Immersive Environments of the as described in \cite{tan2010future}. This new network is termed the Internet of Things. The Internet of Things, aims to seamlessly fuse people places and things across current communications platforms, realising immersive situations that are enabled through the collection of information from embedded sensors and respond by acting upon corresponding embedded actuators. 

Sensors embedded within our environment range from simple sensors such as temperature, humidity, light intensity and occupation sensors, to location and GPS sensors embedded in mobile devices, telephones and automobiles. Applying approaches such as sensor fusion allows us to emulate even higher level sensor information exposing information that is otherwise not directly observable. All these sensors collect data, and with this data developers can build applications that respond to the given data from the sensors. These could range from an application that can automatically regulate the heating in your home to an application that can tells you which way you should take to work according to the traffic on streets.

In order to enable these behaviors and realise the impending immersive paradigm, sensors of a device need to collect and share context information. Earlier research in the area realised the use of middleware systems for this large scale information provisioning. These included both centralized and decentralized approaches. Centralized approaches such as the IP Multimedia subsystem \cite{Kardeby:2010:UMF:1845879.1846331} and MQTT \cite{HunkelerTS08}, exist as web service portals on the Internet, providing a point of connection for entities to provision context information on an Internet of Things. However, these approaches assume the complete availability and reliability systems which are susceptible to DNS errors, denial of service attacks and dynamic IP configuration issues. They are also prone to creating bottlenecks which affect real-time information sharing which means that information accuracy are not guaranteed. This is important for an Internet of Things where real-time is key to creating Immersive Participation Environments. Furthermore, the scalability is limited when using a centralized approach \cite{Kanter539187}. The problem with scalability and DNS availability therfore makes centralised solutions suboptimal.

In order to fix this it was suggested use distributed approaches. These do not rely on DNS, are more scalable and less prone to DOS attacks. Such approaches include MediaSense \cite{Kanter539187} and UBIWARE \cite{osterle2010memorandum} and these have both the \emph{server} and \emph{client} on one machine. Within the mediasense realization, applications are tied to the platform and are started and terminated with the platform. Every application has its own MediaSense instance and communicates with other devices through this instance.

Another thing that needs to be considered when developing middleware for the concept Internet of Things is that devices in our everyday life has limited resources. The current way applications run on MediaSense, where every application needs its own instance of the platform, makes it very inefficient.