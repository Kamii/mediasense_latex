\section{Goal}
To solve the problem it is required to re-engineer existing platforms towards low resources devices. One such distributed Internet of Things platform is MediaSense. MediaSense in its present form makes it necessary to run the platform once for every application. The way MediaSense works today it uses a lot of resources on the devices it runs on. Every application running on a device needs its own instance of the platform. Because MediaSense is communicating over IP every instance needs its own port open. This means that a user needs to open new ports on the router and firewall for every application running on the device. When several instances of MediaSense is running on a device the network traffic increases, more processing power is used and more memory is used. 

Can a decentralized Internet of Things platform be run as a service or daemon for several applications. Can this reduce the usage of resources. The main goal is to reduce the resource overhead when running MediaSense platform on devices. Can we design and implement a more reliable architecture, that improves performance, reduces the resource overhead and provides a common service for all applications running on one device? Can this be implemented with Java as programming language without extensive architectural changes?