\section{Problem}
\label{problem}
The Internet of Things aims to enable massive immersive applications through the use of context information from computers and smartphones. Since the sale of smartphones surpassed that of computers and laptops in 2011 \cite{canalsys}, the Internet of Things will heavily incorporate more ubiquitous devices with lower resource availability that can be mass deployed. Therefore, these applications must be able to run on ubiquitous devices.

Current Internet of Things middleware are not aimed towards resource constrained devices where they can run on ubiquitous devices, this is not designed to be efficient on such devices. One of the main requirements for Internet of Things middleware defined by Theo Kanter et al. \cite{Kanter539187} is for it to be lightweight and resource efficient. The current design of distributed Internet of Things middleware causes a resource overhead for each application which in turn excludes usage of multiple applications on such ubiquitous devices. 

By improving the resource usage of distributed Internet of Things middlewares, they will become usable on ubiquitous devices. Allowing several applications to share necessary resources can reduce the resource overhead. 


