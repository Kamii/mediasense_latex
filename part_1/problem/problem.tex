\section{Problem}
The Internet of Things aim to enable massive immersive applications. Since the sale of smartphones surpassed that of computers and laptops in 2011 \cite{canalsys} the Internet of Things will heavily incorporate more ubiquitous devices with lower resource availability that can be mass deployed. Therefore, these applications must be able to run on ubiquitous devices.

Distributed Internet of Things middlewares are not designed for resource constrained devices. The MediaSense platform has been implemented as an Android application but this version of the platform only runs one simple test application \cite{mediasenseweb}. The platform can be run on ubiquitous devices but is not designed to be efficient on such devices. One of the main requirements for Internet of Things middleware defined by Theo Kanter et al. \cite{Kanter539187} is for it to be lightweight and resource efficient. The current design of distributed Internet of Things middleware causes an resource overhead for each application which in turn excludes usage of multiple applications on such ubiquitous devices. 

By improving the resource usage of distributed Internet of Things middlewares they will become usable on ubiquitous devices. Allowing several applications to share necessary resources can reduce the resource overhead. 
