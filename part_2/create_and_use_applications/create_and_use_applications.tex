\section{Applications Using Context Information}
To build this big network of context aware applications that interact with users without them knowing it, some conditions must be understood. Because of the mobility of the devices the context information need to be collected and shared through wireless network. Moreover, the infrastructure for an Internet of Things platform have to scale well to increase amounts of users and have to always be available to these users \cite{Kanter539187}.

Another condition to consider is the large number of applications that should be able to run on a single ubiquitous device. The ubiquitous devices have limited performance, even if the evolution of the hardware is going forward. Due to their size it is important for an application to be lightweight. Ubiquitous computer devices have limited processing capability, small memory space and have limited battery time. The capability of processing is limited which make them not well suited for computation of intensive tasks. A ubiquitous device also has limited amount of available memory. This two conditions makes it more important to have lightweight applications and services on ubiquitous devices. As mentioned before battery time is also limited, for example the battery time is decreased whenever the device has network communication. Therefore it is important to make the applications and services efficient in the use of network communication. 
If we run a context-aware application on a Raspberry Pi we need to consider that the device only has 512 MB RAM and that this device should be able to run several instances of similar applications so the footprint from the network communication need to be reduced. 

The applications also need near instant access to context-information. Context-information need to be accessed in real-time to provide updated data from sensors. This cannot be provided with a centralized solution \cite{TheMediaSenseFramework}. Real-time delivery of context-information between endpoints is important to existing and future mobile applications. 
