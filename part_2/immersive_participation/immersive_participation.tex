\section{Immersive Participation}
Immersive participation is focused on participation on the Internet via ubiquitous computing and context-awareness. It enables people, places and things to connect to each other to create Immersive Participation Environments. Immersive Participation Environments provides users with context-awareness everywhere which makes the users participate as if they are in a virtual world with places, things and people in it. Common examples of Immersive Participation Environments today include Google Ingress \cite{ingress} where users join teams and compete with other teams in a virtual world where they need to take over real world artefacts, TURF \cite{turf} where users capture real world places and gain points and RATS Theatre's \cite{rats} application called Maryam \cite{maryam} which is an interactive theater where audio clips is triggered depending on the user's Global Positioning System (GPS) location.

Larger scale immersive applications will benefit from scalable distributed information sharing and also remove bottlenecks and dependencies on centralized web portals on the internet.
The way humans interact with each other and things around them will change when sensor information can be shared and accessed ubiquitously. Creating immersive environments that blend the natural world with a seamless internet of things require that we are able to understand the situation of the users in real-time this understanding is termed context awareness.
