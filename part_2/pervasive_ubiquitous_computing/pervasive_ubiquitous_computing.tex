\section{Pervasive \& Ubiquitous Computing}
Pervasive and Ubiquitous computing describes the philosophy of "everything everywhere computing" and defines the concept of having small computers everywhere. To build context-aware applications contextual data is needed and the ubiquitous computing paradigm can be used to make collection of this data possible.
Computers today are everywhere, they can be found in phones, TVs, cars, kitchen machines, watches et cetera. Jens Malmodin et al. \cite{fehske2011global} predicts that 50 billion devices will be connected together by 2020. As computers became available for personal use it was important to make them easy to use. When computers become pervasive and ubiquitous it becomes important to make the computers themselves smart and easy to use instead. According to Mark Weiser who is the founder of the concept Ubiquitous computing the goals of Ubiquitous computing can be summarized as: 

\begin{quotation}
Ubiquitous computing has as its goal the enhancing computer use by making many computers available throughout the physical environment, but making them effectively invisible to the user. \cite{237456}
\end{quotation}

With the concept of ubiquitous computing the many computers in our surroundings should always be ready to deliver their services to the user without making the user aware that he is interacting with it. Thus devices become \emph{invisible}, assisting the users without getting in the way or requiring the users to manually interact with them. 

According to The Swedish Data Inspection Board \cite{datainspect}, ubiquitous computing is a new computer era. In the past one human only had one computer, so called personal computer (PC). In the last years this have been changing. Computers have been integrated into artefacts that humans use in their everyday life. For example if you buy new running shoes you can find a computer chip in them that collects data about your running. Almost everything we use in our everyday life have some kind of computer in it. With ubiquitous computing places and physical objects will be connected to each other and communicate with each other and humans. Because of this the new computer era, ubiquitous computing is growing and one person is using many computers, without knowing that they exist. 

If computers will be everywhere, they must be as small as possible. They also need to be cheap and be low-power computers \cite{datainspect}. A example of such computers is Raspberry Pi, a credit card sized computer and can be run with 4 x AA batteries. Other examples is smartphones like Samsung Galaxy 3 and tablets like Nexus 7. According to Moore's Law the coming years devices will be smaller, more powerful and they will be much cheaper \cite{591665}. This gives a good chance that computers which already are embedded in all things around us will be powerful enough to run multiple context aware applications. These kinds of devices make it possible for the ubiquitous computing paradigm to grow and smart solutions for collecting context information and share it with others will be needed.