\section{Pervasive \& Ubiquitous Computing}
Pervasive and Ubiquitous computing describes the philosophy of "everything everywhere computing". To build context-aware applications context information is needed and ubiquitous computing can be used to make collection of this data possible.
Computers today are everywhere; in phones, TVs, cars, kitchen machines, watches, etc. Jens Malmodin et al. \cite{fehske2011global} predicts that 50 billion devices will be connected together by 2020. When computers become pervasive and ubiquitous it becomes important to make the computers themselves smart and easy to use instead. According to Mark Weiser in \cite{237456} the goal of Ubiquitous is 

\begin{quotation}
\centering
[...] the enhancing computer use by making many computers available throughout the physical environment, but making them effectively invisible to the user.
\end{quotation}

According to The Swedish Data Inspection Board \cite{datainspect}, ubiquitous computing is a new computer era. In the past one human only had one computer, so called personal computer (PC). In the last years this have been changing. Computers have been integrated into artefacts that humans use in their everyday life. Shoes can contain a computer chip that collects data about your running \cite{saponas2006devices}. With ubiquitous computing places and physical objects will be connected and communicate with each other and humans. 

If computers will be everywhere, they must be as small as possible. They also need to be cheap and be low-powered computers \cite{datainspect}. An example is the Raspberry Pi, a credit card sized computer and can be run with 4 x AA batteries. According to Moore's Law the coming years devices will be smaller, more powerful and they will be much cheaper \cite{591665}. This gives a good chance that computers which already are embedded in all things around us will be powerful enough to run multiple context aware applications.
