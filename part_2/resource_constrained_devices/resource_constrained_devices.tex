%\section{Tendency Towards Resource Constrained Devices}
Alan Messer et al \cite{alanmesser} have identified a problem with the concept of pervasive computing. Peoples vision is to execute a service on any device without worrying about whether the service is designed for the device. Alan Messer et al suggests that it will be difficult to create a service that can be run on all devices due to the resource constraints on different devices. Devices processing, memory, network, power capacities differs from each other and services need to be tailored to fit on different devices. 

A lot of existing computer software was first developed for computers with high resources. When software runs on a computer with low resources it needs a redesign to fit on the device. The operating system Android is built on Linux which was an operating system that first was used on personal computers. Android Inc redesigned the operating system to fit it on smartphones. The same goes for Microsoft that has developed a version of Windows for tablets and smartphones. Applications like Netflix, Google Chrome and Utorrent have been redesigned to fit different devices. Software can sometimes be run on a device for which it wasn't designed to run on, but the software is not tailored for those devices and will therefore be less efficient causing reduced performance.
