\section{Shared Resources to Reduce Resource Costs}
The resources of computers can be used in an effective way if the applications have resource consuming components located in a shared service. A shared service can act like a container where components like databases, calculations and network communication can be shared between all processes. A common type of application using shared resources is the web browser. In the late 1990s and early 2000s web browsers could only visit one web page and to view multiple web pages simultaneously new windows had to be opened. Tabbed browsing or TDI (tabbed document interface) of web pages was popularized by Mozilla FireFox in 2003 and made it possible to have several web pages opened in one window. Tabbed browsing allows web browsers share services between the tabs displaying web pages. Examples of such services in web browsers are JavaScript engines, rendering engines, plug-ins, extensions, et cetera. 
MySQL has a shared service for all databases running on a server, a so-called daemon, which handles access to all databases on the server. For example, the communication with MySQL is handled by the shared service which listens to a TCP port. When the service gets a request to this port it forwards it to the specified database. If MySQL instead had one service for every database, one network layer is needed for every database which increase the resource usage. 

The tendency to gather resource consuming processed in one place and share access to them between many processes is the basis of the client-server model. Examples of such behaviours are numerous and prominent within the world wide web. Google's massive indexed database of web pages can be queried by millions of users at a time. To handle this Google has large and powerful datacenters to accommodate the users instead of each user downloading a copy of Google's database and querying it locally. Spotify's large collection of music can be accessed by users and streamed to a client giving access to approximately 80 terabytes (20 million songs at an average size of 4MB) of music for which the storage costs alone would be unfeasible for the average user by today's standards.
