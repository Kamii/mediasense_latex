\section{Research Method For Define Requirements}
Due to there only being one stakeholder with sufficient knowledge of MediaSense and as such quantitative methods are not applicable for defining the requirements of the artefact.
The qualitative methods we considered were observation, case study, interview and group discussions. To perform an observational study we would need a subject to observe in its natural environment but distributed Internet of Things middleware is still a subject of research and as such the intended environment does not yet exist. Conducting group discussions is not applicable in our situation because there only is one stakeholder.
Interviews at first seemed like a good approach because there only is one very involved stakeholder with whom we can conduct the interviews and thus get a deep understanding of the stakeholders needs. A problem with interviews is that the reliability or validity of the answers isn't guaranteed \cite{golafshani2003understanding}, even though the respondent might answer to their best ability it is possible that not all requirements are immediately unearthed due to the restricted perspectives. Interviews also tend to stifle creativity and are dependent on the questions asked \cite{johannesson2012design} thus resulting in important requirements being missed. Because the stakeholder also was responsible for the artefact the answers given in an interview could be coloured by his own view of the artefact.
A case study with both interviews and a deeper study of the artefact was then the best solution as this would make it possible to determine how aspects of the artefact worked before conducting unstructured and open-ended interviews to determine how the solution should work.

