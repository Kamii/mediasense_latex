\section{Requirements Definition}
\subsection{Strategy and Method Choice}
The research strategies considered were surveys, case study and action research. As the research strategy for defining requirments, surveys were considered. Surveys are good to use when researchers want to investigate the needs and wants of many stakeholders \cite{johannesson2012design}. MediaSense is a research project still in progress, and because of this there are few people with knowledge of how it works. This means there are not many stakeholders to send the surveys to and, therefore, this strategy was not used. Because of this quantitative research strategies are not applicable for defining the requirements of the artefact. 
The first qualitative research strategy considered for defining requirements was action research. Action research is a research strategy which is good to use when the researchers knowledge of the problem exceeds that of the stakeholders and the stakeholders have limited understanding of the new artefact \cite{johannesson2012design}. This is not the case for this project since the lead developer has broad knowledge about the problem and a good understanding of the new artefact that is going to be developed. Therefore, Action research is not chosen for this action in design science. A case study consisting of both interviews and a deeper study of the existing artefact was the best choice of research strategy. This approach would make it possible to determine how aspects of the artefact worked before conducting unstructured and open-ended interviews with the lead developer of MediaSense to determine how the solution should work. A problem with only using interviews is that the reliability or validity of the answers isn't guaranteed \cite{golafshani2003understanding}, even though the respondent might answer to their best ability it is possible that not all requirements are immediately unearthed due to the restricted perspectives. Interviews also tend to stifle creativity and are dependent on the questions asked \cite{johannesson2012design} thus resulting in important requirements being missed. Because the lead developer was responsible for the artefact, the answers given in an interview could be coloured by his own view of the artefact. Therefore, a deeper study of the existing artefact was done to complete the interviews. The study will involve both reading the code and benchmarking the software.

Alternative research methods observation study and group discussion were considered. To perform an observational study, the researchers would need a subject to observe in its natural environment, but distributed Internet of Things middleware is still a subject of research and as such the intended environment does not yet exist. Conducting group discussions is not applicable in our situation because there is only one lead developer that can participate. 

%\subsection{Research Question}
%As the authors discuss in \cite{johannesson2012design} a research question was formulated to form the basis of the problem explication. For the problem explication part of the project the research questions is the following: 
%
%\begin{quotation}
%What are the major issues with current implementations of distributed Internet of Things middleware which make them unable to realize the Internet of Things vision?
%\end{quotation}

\subsection{Method Application}
The case study of MediaSense began by examining its architecture. The problem explication made it clear that the platform needed to be run as one separate instance. This requires giving applications an interface through which they are able to access the platform. This required knowledge of what parts of MediaSense the applications had access to and deciding which parts of MediaSense should be shared between applications. Interviews with a stakeholder, the lead developer of MediaSense, was conducted to determine the requirements for the artefact. The interviews were connected to the case study where investigations of different scenarios were done. These were done in the form of informal open-ended interviews with the goal of understanding which the problems were with the old artefact and how the new artefact should work.
From the information collected from the case study and the open-ended discussions, a list of requirements was extrapolated. This activity was done in an iterative way under the design and develop activity as new issues were detected. These issues were brought to the stakeholders attention and discussed in order to define new requirements. The MediaSense platform was profiled to obtain a benchmark for the memory consumption before our redesign.
