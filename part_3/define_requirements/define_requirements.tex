\subsection{Defining Requirements}
The research strategies considered were surveys, case study and action research. Surveys was discussed to used as research strategy for defining requirements. Surveys is good to use when researchers want to investigate the needs and wants of many stakeholders \cite{johannesson2012design}. MediaSense is a research project still in progress and because of this there are few people with knowledge of how it works. This means there are not many stakeholders to send the surveys to and therefore this strategy was not used. Because of this quantitative research strategies are not applicable for defining the requirements of the artefact. 
Looking instead at qualitative research strategies, action research was first considered as a research strategy for defining requirements. Action research is a research strategy which is good to use when the researchers knowledge exceeds that of the stakeholders and the stakeholder have limited understanding of the new artefact \cite{johannesson2012design}. This is not the case for this project since the lead developer has broad knowledge about the problem and a good understanding of the new artefact that is going to be developed. Therefore, Action research is not chosen for this action in design science. 
A case study as research strategy with both interviews and a deeper study of the existing artefact was the best solution as this would make it possible to determine how aspects of the artefact worked before conducting unstructured and open-ended interviews to determine how the solution should work. A problem with only using  interviews is that the reliability or validity of the answers isn't guaranteed \cite{golafshani2003understanding}, even though the respondent might answer to their best ability it is possible that not all requirements are immediately unearthed due to the restricted perspectives. Interviews also tend to stifle creativity and are dependent on the questions asked \cite{johannesson2012design} thus resulting in important requirements being missed. Because the lead developer also was responsible for the artefact the answers given in an interview could be coloured by his own view of the artefact. Therefore a deeper study of the existing artefact was done to complete the interviews. The study will involve both reading the code and benchmarking the software.

Alternative research methods observation study and group discussion were considered. To perform an observational study the researchers would need a subject to observe in its natural environment but distributed Internet of Things middleware is still a subject of research and as such the intended environment does not yet exist. Conducting group discussions is not applicable in our situation because there is only one lead developer that can participate. 
Interviews at first seemed like a good approach because there only is one very involved developer whom we can conduct the interviews and thus get a deep understanding of the lead developers needs. 

