\section{Defining Requirements}
\subsection{Strategy and Method Choice}
The research strategies considered were surveys, case study and action research. Surveys was discussed to used as research strategy for defining requirements. Surveys is good to use when researchers want to investigate the needs and wants of many stakeholders \cite{johannesson2012design}. MediaSense is a research project still in progress and because of this there are few people with knowledge of how it works. This means there are not many stakeholders to send the surveys to and therefore this strategy was not used. Because of this quantitative research strategies are not applicable for defining the requirements of the artefact. 
Looking instead at qualitative research strategies, action research was first considered as a research strategy for defining requirements. Action research is a research strategy which is good to use when the researchers knowledge of the problem exceeds that of the stakeholders and the stakeholders have limited understanding of the new artefact \cite{johannesson2012design}. This is not the case for this project since the lead developer has broad knowledge about the problem and a good understanding of the new artefact that is going to be developed. Therefore, Action research is not chosen for this action in design science. 
A case study as research strategy with both interviews and a deeper study of the existing artefact was the best solution as this would make it possible to determine how aspects of the artefact worked before conducting unstructured and open-ended interviews with the lead developer of MediaSense to determine how the solution should work. A problem with only using interviews is that the reliability or validity of the answers isn't guaranteed \cite{golafshani2003understanding}, even though the respondent might answer to their best ability it is possible that not all requirements are immediately unearthed due to the restricted perspectives. Interviews also tend to stifle creativity and are dependent on the questions asked \cite{johannesson2012design} thus resulting in important requirements being missed. Because the lead developer also was responsible for the artefact the answers given in an interview could be coloured by his own view of the artefact. Therefore a deeper study of the existing artefact was done to complete the interviews. The study will involve both reading the code and benchmarking the software.

Alternative research methods observation study and group discussion were considered. To perform an observational study the researchers would need a subject to observe in its natural environment but distributed Internet of Things middleware is still a subject of research and as such the intended environment does not yet exist. Conducting group discussions is not applicable in our situation because there is only one lead developer that can participate. 

\subsection{Research Question}
As the authors discuss in \cite{johannesson2012design} a research question was formulated to form the basis of the problem explication. For the problem explication part of the project the research questions is the following: 

\begin{quotation}
What are the major issues with current implementations of distributed Internet of Things middleware which make them unable to realize the Internet of Things vision?
\end{quotation}

\subsection{Method Application}
To explicate the problem a document study of previous publications regarding distributed Internet of Things and middleware for it was done. Some of the publications regarding MediaSense were provided by the stakeholder \cite{TheMediaSenseFramework}, \cite{Kanter539187}, \cite{Walters413794}. These provide background on Internet of Things, what MediaSense is and how it works theoretically. Publications about Internet of Things and the surrounding theories were found by searching IEEE Xplore, ACM Digital library and Google Scholar. The main search queries used were \emph{Internet of Things}, \emph{Internet of Things middleware}, \emph{Internet of Things centralized} and \emph{Internet of Things decentralized}. Because the concept Internet of Things is a popular research area, a lot of articles was found. Publications was selected by reading the abstract to see if the publication were able to expand the knowledge base for the problem, searches for keywords were done and ranking publications after number of citations was done to narrow down the search result. 

After getting a grasp of the current state of Internet of Things middleware unstructured interviews were conducted with the lead developer of the MediaSense platform. This respondent has a lot of experience with distributed computing and a good understanding of how Internet of Things middlewares works, therefore this respondent was the best person to interview. The interviews were conducted in an informal manner in the respondent's office and in a conference room with access to a whiteboard. No recording or notes were done for the interviews thus availed both authors to be active in the interviews and discussions could be done without thinking of recording or taking notes. These interviews were conducted iteratively and in parallel with a document study of the existing source code of MediaSense. This document study yielded questions for upcoming interviews and the interviews in turn gave more information about the inner workings of Internet of Things middlewares and how to proceed the study of MediaSense's source code.

Interviews began with trying to discover the problematics of the chosen Internet of Things middleware. The lead developer of MediaSense first presented a problem that had been identified with MediaSense, a large resource overhead when running multiple applications. The first set of interview questions served to fill the gaps that documentation normally would. The questions aimed to explore how MediaSense worked and to get a broader knowledge base. After studying the source code and getting an understanding of the architecture, the interview questions explored certain modules functionality and how they worked. When the interviewee responded to the questions in an uncertain manner, the follow up questions were formulated to ascertain how they were supposed to work or how it was intended they should work.

After the application of methods used to explicate the problem an analysis was done on the result from the interviews and the document study. This was done by putting together the result from the interviews and the knowledge that was built when the document study of code was done. By using the answers from the interviews and reproducing the problems mentioned by the respondent on a developer instance of MediaSense, the problem was clearer from a technical view.
