\section{Design and Develop Artefact}
With the information gathered in the earlier stages of the design science process, architectural changes will need to be designed. For this process, participative modelling \cite{johannesson2012design} and document study will be used. The participative modelling will mainly be done by drawing architectural models on a whiteboard and the document studies will be done to research similar solutions and usable technologies. 

The development process will be done using Agile Software Development Techniques, especially Scrum's daily meetings \cite{kniberg2007scrum} and Pair Programming \cite{williams2000all}. 
Pair programming is a practice from the software development methodology extreme programming. According to  \cite{williams2000all} two programmers working together will find twice as many solutions to a problem compared to working alone and bugs will be found at an earlier stage which this will give a higher quality artefact. The development will also include a study of the tools and techniques used to develop the artefact. To keep the timeline and give updates to the lead developer of MediaSense weekly meetings will be arranged where the progress of designing and developing the artefact will be presented. 

%flyttat efter iskra feedback

The choice of Agile techniques was motivated by the iterative way the problem explication and requirements collection processes need to be carried out. A waterfall like development process, where all requirements are collected before starting development, was considered but would not have allowed us define the requirements iteratively. This development method could result in an artefact that does not achieve the goals.
The iterative development cycle will begin with the requirements which will be split into smaller tasks that then are added to a project management tool. When functions have been implemented, new functions from the backlog will be chosen and then developed. One day every week a meeting will be held with the stakeholder, where an update on the development progress will be given. These meetings will allow problems or issues encountered during the week to be highlighted, and may also lead to new requirements being defined. From these requirements, new functionality will be extrapolated and added to the feature backlog. 
