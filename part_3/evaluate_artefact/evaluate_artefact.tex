\section{Research Method For Evaluate Artefact}
To evaluate the artifact we will need to validate that the requirements have been met. The strategies considered for evaluation are Surveys, Experiments, Case studies, Ethnography, Theoretical Analysis. 
Doing an ethnographic study would give valuable insights into how an Internet of Things platform would be used and how our redesign would impact usage. Given that Internet of Things still isn't a widely adopted paradigm there would be no precedent to compare cultural impacts to. Such a study would only contribute to the understanding of how people use the Internet of Things and not our artifact specifically.
A case study allows for a deep study of the artefact but can be biased by the researchers perceptions, thus doing a case study of an artifact we ourselves developed will be inconclusive as to if the artifact fulfils the requirements.
Experiments allow us to set up an artificial scenario similar to that in the demonstration. The experiment will be designed to specifically validate all the requirements. A drawback to using experiments is that the artificial scenario doesn't reflect a real life scenario. To rectify this we will use Theoretical analysis of the results from the experiment.
Thus we chose to evaluate the artifact with experiments and theoretical analysis.
