\section{Research Method For Explicate Problem}
To explicate the problem we chose to do a document study of previous publications regarding internet of things middleware. The document study also consisted of reading the source code for MediaSense to get an insight into how MediaSense is constructed and how it worked. Combining this method with interviews of a person with more knowledge about the concept internet of things and knowledge about MediaSense helped to explicate the problem and give a broader knowledge base of the surrounding concepts for internet of things middleware. With a broader knowledge base the problem can then be broken down into a number of subproblems. 

An alternative to these methods could be survey using questionnaires, in which predefined written questions can be asked to a respondent. This method can easy be distributed to a large number of respondents. It was not chosen because the researcher cannot ask follow up questions which makes it hard to discuss a problem situation and because there are few persons with knowledge about MediaSense the result of a questionnaire would not explicate the problem, especially when follow up questions is not possible. If there was a larger group of stakeholders for MediaSense a survey could have been used to explicate the problem, but this was not the case and surveys was excluded.

