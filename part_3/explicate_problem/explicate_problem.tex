\subsection{Explicating The Problem}
To explicate the problem a case study of MediaSense is done. The research methods for this case study is document study of previous publications regarding internet of things middleware. The document study also consisted of reading the source code for MediaSense to get an insight into how MediaSense is constructed and how it worked. Combining this method with interviews of a person with more knowledge about the concept internet of things and knowledge about MediaSense helped to explicate the problem and give a broader knowledge base of the surrounding concepts for internet of things middleware. With a broader knowledge base the problem can be broken down into a number of subproblems. 

An alternative to the research strategy case study could be survey using questionnaires, in which predefined written questions can be asked to a respondent. This method can easy be distributed to a large number of respondents. It was not chosen because the researcher cannot ask follow up questions which makes it hard to discuss a problem situation and because there are few persons with knowledge about MediaSense the result of a questionnaire would not explicate the problem, especially when follow up questions is not possible. If there was a larger group of stakeholders for MediaSense a strategy like survey could have been used to explicate the problem, but this was not the case and surveys were excluded.

Also action research was thought about to use for explicating the problem. Action research was not used though the researchers does not have any experience of the problem and can therefore not offer fresh perspectives to the problem to explicate it. The lead developer of MediaSense have a broad knowledge about the problem and interview can be done when a case study of existing middleware is done. After collecting data for the explicated problem an analysis is done by discussing the problems the researchers found in the source code with the lead developer to confirm the explicated problem. 


