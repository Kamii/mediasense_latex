\section{Explicating the Problem}
\subsection{Strategy and Method Choice}
Action research was considered for explicating the problem. Action research was not used because the researchers does not have any experience of the problem and can not offer fresh perspectives of the problem to explicate it. An alternative research strategy considered was a survey. This strategy can easy be distributed to a large number of respondents by using questionnaires, in which predefined written questions can be asked to a respondent or by doing structured interviews with a smaller group of respondents. It was not chosen because the researcher cannot ask follow up questions which makes it hard to discuss a problem situation and also because there are few persons with knowledge about MediaSense. Therefore, the result of a questionnaire would not explicate the problem, especially when follow up questions are not possible. If there was a larger group of stakeholders for MediaSense a strategy like survey could have been beneficial to explicate the problem, but this was not the case and surveys were excluded.

To explicate the problem a case study of MediaSense was chosen. The research methods for this case study is document study and interviews. The document study will cover both previous publications regarding internet of things middleware and the MediaSense source code to get an insight into how MediaSense is constructed and how it works. The lead developer of MediaSense has in-depth knowledge of the problem and interviews can be held when doing a case study of existing middleware. Combining the study of the source code with interviews with a person with more knowledge about the concept internet of things and knowledge about MediaSense helps to explicate the problem and to aquire a broader knowledge base of the surrounding concepts for Internet of Things middleware. With a broader knowledge base the problem can be broken down into a number of subproblems. After collecting data for the explicated problem the data will be analyzed by finding patterns where the problem occurs and then discussing the problems the researchers found in the source code with the lead developer to confirm the explicated problem. 

\subsection{Research Question}
As the authors discuss in \cite{johannesson2012design} a research question was formulated to form the basis of the problem explication. For the problem explication part of the project the research questions is the following: 

\begin{quotation}
What are the major issues with current implementations of distributed Internet of Things middleware which make them unfit for running on ubiquitous devices?
\end{quotation}

\subsection{Method Application}
To explicate the problem a document study of previous publications regarding distributed Internet of Things middleware for it was done. Some of the publications regarding MediaSense were provided by the stakeholder \cite{TheMediaSenseFramework}, \cite{Kanter539187}, \cite{Walters413794}. These provide background on Internet of Things, what MediaSense is and how it works theoretically. Publications about Internet of Things and the surrounding theories were found by searching IEEE Xplore, ACM Digital library and Google Scholar. The main search queries used were \emph{Internet of Things}, \emph{Internet of Things middleware}, \emph{Internet of Things centralized} and \emph{Internet of Things decentralized}. Because the concept Internet of Things is a popular research area, a lot of articles was found. Publications was selected by reading the abstract to see if the publication were able to expand the knowledge base for the problem, searches for keywords were done and ranking publications after number of citations was done to narrow down the search result. 

After getting a grasp of the current state of Internet of Things middleware unstructured interviews were conducted with the lead developer of the MediaSense platform. This respondent has a lot of experience with distributed computing and a good understanding of how Internet of Things middlewares works, therefore this respondent was the best person to interview. The interviews were conducted in an informal manner in the respondent's office and in a conference room with access to a whiteboard. No recording or notes were done for the interviews thus availed both authors to be active in the interviews and discussions could be done without thinking of recording or taking notes. These interviews were conducted iteratively and in parallel with a document study of the existing source code of MediaSense. This document study yielded questions for upcoming interviews and the interviews in turn gave more information about the inner workings of Internet of Things middlewares and how to proceed the study of MediaSense's source code.

Interviews began with trying to discover the problematic of the chosen Internet of Things middleware. The lead developer of MediaSense first presented a problem that had been identified with MediaSense, a large resource overhead when running multiple applications. The first set of interview questions served to fill the gaps that documentation normally would. The questions aimed to explore how MediaSense worked and to get a broader knowledge base. After studying the source code and getting an understanding of the architecture, the interview questions explored certain modules functionality and how they worked. When the interviewee responded to the questions in an uncertain manner, the follow up questions were formulated to ascertain how they were supposed to work or how it was intended they should work.

After the application of methods used to explicate the problem an analysis was done on the result from the interviews and the document study. This was done by putting together the result from the interviews and the knowledge that was built from the document study. By using the answers from the interviews and reproducing the problems mentioned by the respondent on a developer instance of MediaSense, the problem was clearer from a technical view. To confirm the problems the researchers found new interviews was done with the lead developer. The researchers view of the problem was discussed in this interview to confirm the explicated problem.
