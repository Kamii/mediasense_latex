\chapter{Method}
To be able to solve the problem it is necessary to rethink and redesign core-components of the an existing middleware. Using traditional quantitative or qualitative research methods would not involve any actual designing. These research methods produce data based on past or present conditions and don't allow researchers to develop artefacts.
Design science research differs in one major way from qualitative or quantitative empirical studies. Where empirical studies focus on finding patterns in the past or present of an industry or a field, design science is intended to design and develop new solutions to actual problems \cite{bider2012design} and aims to improve and create artefacts that can improve the world \cite{johannesson2012design}. 
Design science is well suited for this problem because it involves a step where requirements are identified, a design and develop action where the researchers develop an artefact based on the requirements and an evaluation action where said artefact is evaluated to validate the requirements have been met and thus if the artefact solved the problem. Consequently design science is a good method to use for this problem where a redesign of an existing middleware for the Internet of Things is needed.

\begin{figure}[h!]
	\centering
    	\includegraphics[scale=0.50]{part_3/design_science.png}
		\caption{Diagram of the Design Science Method \cite{johannesson2012design}} 
		\label{ds}
\end{figure}

The method chosen for this research is design science as defined by Johannesson and Perjons in A Design science primer \cite{johannesson2012design}. It is well structured and has guidelines that makes it easy to use as opposed to other definitions of design science such as the one laid out in Memorandum on design-oriented information systems research \cite{osterle2010memorandum} or the one mentioned in Design science in information systems research \cite{hevner2004design}. Johannesson and Perjons mention that Design Science consists of 5 actions, as shown in figure \ref{ds}. This actions not proposed to be used in a sequence, the steps should be used in an iterative way. In design science it is possible to use other methods to answer questions about artefact. Every action in \ref{ds} should have its own methodology or it can even have several methodologies.


\section{Research method for Explicate Problem}
To explicate the problem we chose to do a document study of previous publications regarding internet of things middleware. The document study also considered of reading the source code for MediaSense to get an insight into how MediaSense is constructed and how it worked. Combining this method with interviews of a person with more knowledge about the concept internet of things and knowledge about MediaSense helped to explicate the problem and give a broader knowledge base of the surrounding concepts for internet of things middleware. With a broader knowledge base the problem can then be broken down into a number of subproblems. 

An alternative to these methods could be survey using questionnaires, in which predefined written questions can be asked to a respondent. This method can easy be distributed to a large number of respondents. It was not chosen because the researcher cannot ask follow up questions which makes it hard to discuss a problem situation and because there are few persons with knowledge about MediaSense the result of a questionnaire would not explicate the problem, especially when follow up questions is not possible. If there was a larger group of stakeholders for MediaSense a survey could have been used to explicate the problem, but this was not the case and surveys was excluded.

\subsection{Defining Requirements}
The research strategies considered were surveys, case study and action research. Surveys was discussed to used as research strategy for defining requirements. Surveys is good to use when researchers want to investigate the needs and wants of many stakeholders \cite{johannesson2012design}. MediaSense is a research project still in progress and because of this there are few people with knowledge of how it works. This means there are not many stakeholders to send the surveys to and therefore this strategy was not used. Because of this quantitative research strategies are not applicable for defining the requirements of the artefact. 
Looking instead at qualitative research strategies, action research was first considered as a research strategy for defining requirements. Action research is a research strategy which is good to use when the researchers knowledge exceeds that of the stakeholders and the stakeholder have limited understanding of the new artefact \cite{johannesson2012design}. This is not the case for this project since the lead developer has broad knowledge about the problem and a good understanding of the new artefact that is going to be developed. Therefore, Action research is not chosen for this action in design science. 
A case study as research strategy with both interviews and a deeper study of the existing artefact was the best solution as this would make it possible to determine how aspects of the artefact worked before conducting unstructured and open-ended interviews to determine how the solution should work. A problem with only using  interviews is that the reliability or validity of the answers isn't guaranteed \cite{golafshani2003understanding}, even though the respondent might answer to their best ability it is possible that not all requirements are immediately unearthed due to the restricted perspectives. Interviews also tend to stifle creativity and are dependent on the questions asked \cite{johannesson2012design} thus resulting in important requirements being missed. Because the lead developer also was responsible for the artefact the answers given in an interview could be coloured by his own view of the artefact. Therefore a deeper study of the existing artefact was done to complete the interviews. The study will involve both reading the code and benchmarking the software.

Alternative research methods observation study and group discussion were considered. To perform an observational study the researchers would need a subject to observe in its natural environment but distributed Internet of Things middleware is still a subject of research and as such the intended environment does not yet exist. Conducting group discussions is not applicable in our situation because there is only one lead developer that can participate. 
Interviews at first seemed like a good approach because there only is one very involved developer whom we can conduct the interviews and thus get a deep understanding of the lead developers needs. 


\section{Design and Develop Artefact}
With the information gathered in the earlier stages of the design science process, architectural changes will need to be designed. For this process, participative modelling \cite{johannesson2012design} and document study will be used. The participative modelling will mainly be done by drawing architectural models on a whiteboard and the document studies will be done to research similar solutions and usable technologies. 

The development process will be done using Agile Software Development Techniques, especially Scrum's daily meetings \cite{kniberg2007scrum} and Pair Programming \cite{williams2000all}. 
Pair programming is a practice from the software development methodology extreme programming. According to  \cite{williams2000all} two programmers working together will find twice as many solutions to a problem compared to working alone and bugs will be found at an earlier stage which this will give a higher quality artefact. The development will also include a study of the tools and techniques used to develop the artefact. To keep the timeline and give updates to the lead developer of MediaSense weekly meetings will be arranged where the progress of designing and developing the artefact will be presented. 

%flyttat efter iskra feedback

The choice of Agile techniques was motivated by the iterative way the problem explication and requirements collection processes need to be carried out. A waterfall like development process, where all requirements are collected before starting development, was considered but would not have allowed us define the requirements iteratively. This development method could result in an artefact that does not achieve the goals.
The iterative development cycle will begin with the requirements which will be split into smaller tasks that then are added to a project management tool. When functions have been implemented, new functions from the backlog will be chosen and then developed. One day every week a meeting will be held with the stakeholder, where an update on the development progress will be given. These meetings will allow problems or issues encountered during the week to be highlighted, and may also lead to new requirements being defined. From these requirements, new functionality will be extrapolated and added to the feature backlog. 

\section{Research Method For Demonstrate Artefact}
A demonstration of the artefact will be done for the stakeholder based on a use case scenario. Because the artefact is a distributed system intended to be used on a large scale and as the artefact still is a research project and the applications intended to be used with it have not been constructed, a real life case cannot be constructed for the demonstration. The scenario will be enacted with a test application and on a smaller scale. The aim of the demonstration will be to cover all requirements and demonstrate that all goals have been met.  

\subsection{Evaluate Artefact}
To evaluate the artifact it is necessary to validate that the requirements gathered in the earlier action \emph{defining requirements}, have been met. The strategies considered for evaluation are Surveys, Experiments, Case studies, Ethnography, Theoretical Analysis. 
Doing an ethnographic study would give valuable insights into how an Internet of Things platform would be used and how our redesign would impact usage. Given that Internet of Things still isn't a widely adopted paradigm there would be no precedent to compare cultural impacts to. Such a study would only contribute to the understanding of how people use the Internet of Things and not our artifact specifically. A case study allows for a deep study of the artefact but can be biased by the researchers perceptions, thus doing a case study of an artifact we ourselves developed will be inconclusive as to if the artifact fulfils the requirements.
Experiments allow us to set up an artificial scenario similar to that in the demonstration. The experiment will be designed to specifically validate all the requirements. A drawback to using experiments is that the artificial scenario doesn't reflect a real life scenario. To rectify this we will use Theoretical analysis of the results from the experiment. Thus we chose to evaluate the artifact with experiments and theoretical analysis.


\section{Ethical Considerations}
All information uncovered in the interviews will be used confidentially, and we will assure the respondent consensually agree to have all answers published in this thesis.
The MediaSense platform uses the GNU Lesser General Public License version 3 \cite{gnu}, and as such, there is no confidentiality we need to observe regarding the source code of it.
All test environments used for testing the distributed Internet of Things middleware will be run on a local sandbox for development so the nodes in the network will only contain our own computers. 
