\subsection{Method Application}
To explicate the problem a document study of previous publications regarding distributed Internet of Things and middleware for it was done. Some of the publications regarding MediaSense were provided by the stakeholder \cite{TheMediaSenseFramework}, \cite{Kanter539187}, \cite{Walters413794}. These provide background on Internet of Things, what MediaSense is and how it works theoretically. Publications about Internet of Things and the surrounding theories were found by searching IEEE Xplore, ACM Digital library and Google Scholar. The main search queries used were \emph{Internet of Things}, \emph{Internet of Things middleware}, \emph{Internet of Things centralized} and \emph{Internet of Things decentralized}. Because the concept Internet of Things is a popular research area, a lot of articles was found. Publications was selected by reading the abstract to see if the publication were able to expand the knowledge base for the problem, searches for keywords were done and ranking publications after number of citations was done to narrow down the search result. 

After getting a grasp of the current state of Internet of Things middleware unstructured interviews were conducted with the lead developer of the MediaSense platform. This respondent has a lot of experience with distributed computing and a good understanding of how Internet of Things middlewares works, therefore this respondent was the best person to interview. The interviews were conducted in an informal manner in the respondent's office and in a conference room with access to a whiteboard. No recording or notes were done for the interviews thus availed both authors to be active in the interviews and discussions could be done without thinking of recording or taking notes. These interviews were conducted iteratively and in parallel with a document study of the existing source code of MediaSense. This document study yielded questions for upcoming interviews and the interviews in turn gave more information about the inner workings of Internet of Things middlewares and how to proceed the study of MediaSense's source code.

Interviews began with trying to discover the problematics of the chosen Internet of Things middleware. The lead developer of MediaSense first presented a problem that had been identified with MediaSense, a large resource overhead when running multiple applications. The first set of interview questions served to fill the gaps that documentation normally would. The questions aimed to explore how MediaSense worked and to get a broader knowledge base. After studying the source code and getting an understanding of the architecture, the interview questions explored certain modules functionality and how they worked. When the interviewee responded to the questions in an uncertain manner, the follow up questions were formulated to ascertain how they were supposed to work or how it was intended they should work.

After the application of methods used to explicate the problem an analysis was done on the result from the interviews and the document study. This was done by putting together the result from the interviews and the knowledge that was built when the document study of code was done. By using the answers from the interviews and reproducing the problems mentioned by the respondent on a developer instance of MediaSense, the problem was clearer from a technical view.

