\subsection{Method Application}
The case study of MediaSense began by examining its architecture. The problem explication made it clear that the platform needed to be run as one separate instance and give applications an interface to communicate with it through. This required knowledge of what parts of MediaSense the applications had access to and deciding which parts of MediaSense should be shared between applications. Interviews with a stakeholder, the lead developer of MediaSense, was conducted to determine the requirements for the artefact. The interviews were connected to the case study where investigations of different scenarios were done. These were done in the form of informal open-ended interviews with the goal of understanding which the problems were with the old artefact and how the new artefact should work.
From the information collected from the case study and the open-ended discussions a list of requirements was extrapolated. This activity was done in an iterative way under the design and develop activity as new issues were detected. These issues were brought to the stakeholders attention and discussed in order to define new requirements. The MediaSense platform was profiled to obtain a benchmark for the memory consumption before our redesign.
