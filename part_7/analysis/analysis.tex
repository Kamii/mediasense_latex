\section{Analysis}
After running the different scenarios the researchers believe that all the requirements was met. The non functional requirements was also met. The code was written with Java 1.5. An object oriented style was followed the decision of using RMI instead of other RPC techniques made this requirements easy to follow. The overlay that was provided in the old artefact was not modified thus the lead developer preferred an unmodified network overlay and this requirement was met by the researchers. All functional requirements was tested with different scenarios as shown in method applications from this chapter. As shown from the memory measurement the platform still uses a lot of memory and if only one applications is running on a device there is no benefits of using the redesigned version of MediaSense. As shown from the test result the benefits of the redesigned version of MediaSense comes when more than one applications is running on a device. In the old version every instance of MediaSense with a small application with basic functionality takes around 60-70 mb memory. If the device is running n applications the minimum memory usage will be n*70. Notice that it is hard to measure how much memory an application is using in the old design because this version was applications invoked. When running the new version of MediaSense on a device with n applications, the background process which is the MediaSense platform with RMI support uses between 60 and 70 MB memory. Also in this case a minimal application with basic functionally is running on the device and it uses around 20 MB memory. This means that n applications memory usages will be 70+n*20 MB memory. With the new design of MediaSense the heavy layers of MediaSense is only necessary to be run once on every device. The conclusions is if more than one application is running on a device the new version of MediaSense will use less memory. 

The CPU time is the amount of time a CPU is used for processing a computer program. This test was done by starting the platforms and connect applications to it and then run it in 5 minutes to see how much processing time the artefact uses. As shown in the table the new version of the artefact uses more processing time when the platform only have one application connected to it. Like the memory usage the benefits of the new artefact comes when more than one application is running on the device. 

When running the new version of MediaSense the number of threads are more in comparison with the old version. The researchers believe this is because more threads is needed when RMI is used. As shown in the graph the new version uses less threads when four applications is connected to the platform. This means that even if the number of threads is more it can be effective to use when several applications is running on a device. 

The new artefact also solves the problem that was explicated before where every running instance of MediaSense needs its own network port to communicate with other nodes in the network. With the common network layer only one port is needed to be open and all applications can use this port to communicate with other nodes. This makes the middleware more user friendly and no configuration is needed to run several applications on devices. 
