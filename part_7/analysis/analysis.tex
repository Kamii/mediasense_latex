\section{Analysis}
After running the different scenarios the researchers believe that all the requirements were met. The non functional requirements were also met. The code was written in Java 1.5. An object oriented style was followed, the decision of using RMI instead of other RPC techniques aided this greatly. The overlay that was provided in the old artefact was not modified because the lead developer of MediaSense preferred an unmodified network overlay. All functional requirements were tested and evaluated. As shown by the memory measurements the platform still uses a lot of memory and if only one application is running on a device there are no benefits of using the redesigned version of MediaSense. The results from the measurements show that the redesigned version of MediaSense uses less memory when more than one application is running on a device. In the old version an instance of MediaSense with a small application with basic functionality takes around 60-70 mb memory. If the device is running n applications the minimum memory usage will be n*70. It is hard to measure how much memory an application is using in the version of MediaSense because the middleware was application invoked and could not be run as a separate process. When running the new version of MediaSense the platform with RMI support uses between 60 and 70 MB memory. Additionally a minimal application with basic functionally running on the device uses around 20 MB memory. On a device running several applications this means that for n applications the new version of MediaSense will only use 70+n*20 MB memory. The new version requires only one instance of the middleware to run per device, making the average memory usage per application lower when more than one application is running.

The CPU time is the amount of time a CPU is used for processing a computer program. This test was done by starting the platforms, connecting applications to it and then run it for 5 minutes to see how much processing time the artefact uses. As shown in the table the new version of the artefact uses more processing time when the platform only has one application connected to it. Like the memory usage the benefits of the new artefact comes when more than one application is running on the device. 

When running the new version of MediaSense the number of threads are closer to the old version than memory or CPU time were. As shown in the graph the new version have fewer threads when four applications are connected to the platform. The slight difference in thread count and memory usage is because RMI requires extra thread to handle the connection between stubs and marshalling of objects. 

The new artefact also solves the explicated problem where every running instance of MediaSense needs its own network port to communicate with other nodes in the network. The reengineered version of MediaSense uses a common network layer and only one open port is needed. All applications connected to the middleware then uses it as a proxy to communicate with other nodes. This makes the middleware more user friendly and no configuration is needed to run several applications on devices. 
