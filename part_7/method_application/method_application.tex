\section{Method Application}
The research strategy used to evaluate the artefact was an experiment. The experiment was done by running an instance of the artefact and measure the memory usage while it is running. First the platform was started to see how much resources the platform use. When the platform was connected to the network, the researchers started to connect applications to the platform and take notes of how much memory every application was using. To test the old version of the middleware a node farm was used. A bash script was used to start several applications where every application had its own MediaSense platform. 

When four applications had been started and connected to the platform a pattern in the resource usage was found. The data was then compared to data from the old artefact were an analysis was done to see if the new version use less memory resources. To see if all requirements were fulfilled different tests were done where every test had an expected result. All results were discussed between the researchers to address if the result was as expected and if the requirements had been met. 

The experiment was done on an PC with operating system Ubuntu 12.04.2 LTS. The computer in use has an Intel Core i3-2350M processor and 8 GB Memory. To measure the resources the applications Gnome System Monitor 3.4.1 \cite{gnomesm} and htop 1.0.1 \cite{htop} was used.

The non-functional requirements were not evaluated with a specific approach, they were discussed as the experiments took place to see that they were addressed and met. To measure if the new artefact consumed less of the devices memory one to four applications were run with both the old version of the middleware and the reengineered version. The results were noted and a comparison was then done. 
